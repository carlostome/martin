\documentclass[11pt,a4paper]{scrartcl}
\usepackage[utf8]{inputenc}
\usepackage[english]{babel}
\usepackage{geometry}
\usepackage{amsmath}
\usepackage{amssymb}
\usepackage{amsthm}
\usepackage{xcolor}
\usepackage{graphicx}
\usepackage{todonotes}
\usepackage{enumitem}  % http://www.ctan.org/pkg/enumitem
\usepackage[noabbrev]{cleveref}
\usepackage{fancyvrb}
\usepackage{unicornsoflove}
\usepackage{hyperref}
\usepackage{bussproofs}

 \DefineVerbatimEnvironment
   {code}{Verbatim}
   {} % Add fancy options here if you like.

\author{Carlos Tomé Cortiñas
  \and Fabian Thorand
  \and Ferdinand van Walree
  \and Renate Eilers
  \and  \small{Department of Information and Computing Sciences, Utrecht University}
  }
\title{Stepwise construction of simple Agda programs}
\subtitle{Research Report}
\date{\today}

\begin{document}
\maketitle

\begin{abstract}
\end{abstract}

\tableofcontents

\section{Program Construction}

\subsection{Assumptions and Restrictions}

\begin{itemize}
\item fully saturated functions (i.e.\ all parameters are occurring on the left
  hand side of the definition)
\item all types are in \texttt{Set}, the lowest level of the universe hierarchy
\item data constructors don't have function arguments
\end{itemize}

\subsection{Rules of Construction}

\paragraph{Case splitting}

\begin{itemize}
\item given a $n$-ary function $f : (x_1 : T_1) \rightarrow (x_2 : T_2\ x_1)
  \rightarrow \cdots \rightarrow T_{n+1}\ x_1\ \ldots\ x_n $ where the argument
  types can depend on the values of the arguments to their left, where
  $T_i\ x_1\ \ldots\ x_{i-1}$ can be a function type, a data type or
  \texttt{Set}. In the latter case, the arguments passed to $T_i$ do not
  actually matter
\item note that it is only possible to match on data types
\item case splits are only allowed in definition clauses of the following form
  \begin{equation}
    \label{eq:f_def}
    f\ p_1\ \ldots\ p_n = \mathbf{hole}
  \end{equation}
  where $p_1, \ldots, p_n$ denote the argument patterns and \textbf{hole} is a
  metavariable. This means that we can only case split, if the whole right hand
  side consists of a single hole.
\item abstract syntax of patterns
  \begin{equation}
    \begin{array}{rcl}
      x & \in & \mathbf{Var} \\
      p & ::= & x \mid \mathbf{Constr}\ p_1\ \ldots\ p_n
    \end{array}
  \end{equation}
\item a case split can only be applied to (possibly nested) variables in one of
  the patterns, if the type of that variable is a data type (i.e.\ it has been
  introduced by a \texttt{data} declaration)
\item a successful case split on a variable $x$ replaces the current clause with
  one new clause for every data constructor corresponding to the type of $x$,
  replacing the occurrence of $x$ with a new pattern and fresh variables for
  each constructor argument.
  Only constructors whose type can be unified with the type of $x$ are
  considered.
  This is demonstrated in the example below.
  \begin{lstlisting}[mathescape, gobble=4, language = agda]
    data Foo : Nat -> Set where
      Bar : Foo 0
      Baz : forall {n} -> Foo (succ n)

    foo : (n : Nat) -> Foo n -> Nat
    foo zero Bar = {!!}
    foo (succ n) Baz = {!!}
  \end{lstlisting}
  Case splitting on the second argument (after the first argument) in the first clause only introduced the
  \texttt{Bar} constructor, as the type of the \texttt{Baz} constructor does not
  unify with \texttt{Foo 0}.
\item splitting variables can affect other patterns, e.g. by introducing dot
  patterns\footnote{\url{http://agda.readthedocs.io/en/latest/language/function-definitions.html\#dot-patterns}}.
  In the example above, this can be witnessed by first splitting on the second
  argument, resulting in

  \begin{lstlisting}[mathescape, gobble=4, language = agda]
    foo : (n : Nat) -> Foo n -> Nat
    foo .0 Bar = {!!}
    foo _ Baz = {!!}
  \end{lstlisting}

\end{itemize}

some heuristics to guide case splitting
\begin{itemize}
\item perform case splits in a breadth-first manner, i.e.\ first try split all
  variables on the current level before trying to perform case splitting on the
  variables nested inside constructor patterns.
\item if all $n$ arguments are independent and every data type consists of at
  most $m$ constructors, case splitting all variables on the top-most level
  results in at most $m^n$ clauses.
  In realistic examples making use of Agda's support for dependent types, we
  expect this number to be a lot smaller.
\item sometimes, it might only be possible to split a variable after first
  having split some other arguments before it. An (admittedly rather contrived)
  example for this behaviour is shown below:

  \begin{lstlisting}[mathescape, gobble=4, language = agda]
    contrived : (m n : Nat) -> n + m == m + n -> m + n == n + m
    contrived m n p = ?
  \end{lstlisting}

  Although trivially solvable using a more general definition of symmetry, in this
  specific instance Agda is unable to pattern match on \texttt{p} because it
  cannot normalise the expressions in the $\equiv$-expressions.
\item a good guess for the order in which to perform case splitting might be the
  topological sorted dependency graph of the argument types, starting with the
  arguments on which no further arguments depend.
  In the type signature

  \begin{lstlisting}[mathescape, gobble=4, language = agda]
    (m n : Nat) -> (p : m == n) -> n == m
  \end{lstlisting}
  a valid topologically sorted sequence would be $m, n, p$, hence we would split
  first on $p$.
  The disadvantage of this approach is that it introduces dot patterns in the
  preceding arguments, where we do not know how to handle them appropriately
\end{itemize}

\paragraph{Refinement}

\begin{itemize}
\item A refinement is the introduction of a (possibly) incomplete term of the
  right type in a hole
\item case splitting is not possible after starting to refine the right hand
  side of a clause. If no solution can be found, backtracking is needed.
\item if a hole requires a function, we can only fill it with a definition in
  scope that has a matching type. For now, we do not support the introduction of
  lambda expressions.
\item a hole that requires a data type can be either refined with a data
  constructor, a variable of the right type, or with function returning a value
  of the required type.
  For constructors and functions, new holes are introduced for every argument.
  If the function is a recursive call, then at least one of the arguments must
  be structurally smaller.
\item search depth must be limited to prevent the search from generating
  infinitely nested data constructors.
\end{itemize}

some heuristics for refinement

\begin{itemize}
\item a variable of a correct type should always be preferred over introducing
  data constructors or function calls, because it clearly leads to the simplest
  solutions
\end{itemize}

\subsection{Combining case split and refinement}

\begin{itemize}
\item every clause of the definition is independent from all others
\item the basic algorithm should always first try to find a solution for the
  right hand side of a clause, and only if that fails, perform another case
  split.
\item we should investigate whether it is possible to retain partial solutions
  between to attempts of solving the same goal, that only differ in some
  additional case splits.
\end{itemize}

\subsection{Introducing Properties}

\begin{itemize}
\item basic properties on some function $f$ should be of the following shape
  \begin{equation}
    \label{eq:prop}
    \forall \vec{x}. f\ t_1\ \ldots\ t_n \equiv t_{n+1}
  \end{equation}
  where $t_1, \ldots, t_{n+1}$ are terms possibly involving some of the
  universally quantified variables.
  This provides simple means for stating relations between the inputs and result
  of a function.
\item investigate whether we can handle universal quantification in properties
\item Before starting the refinement phase, we can check which properties apply
  to the current clause. Then we can use Agda to (partially) normalize the
  property expression, resulting in an expression, that might contain additional
  metavariables. But even these partial results can be used to restrict the set
  of possible solutions to a hole, if the metavariables only occur inside some
  constructors.
\item investigate whether properties can be used for guiding case splits. for
  example, if we consider the property $\forall x. x + 0 \equiv x$, then it
  might be sensible to assume that splitting on the second argument is
  preferred.
\item it seems that universally quantified properties are only helpful during
  the search phase, if the universally quantified arguments are exactly those,
  that are not case-split during the search
\end{itemize}

\subsection{The Strategy}

For a given type signature we use Agda's expressive type system to generate a
model solution. Here, we need to take into account the current partial solution
created by the student, because they might have chosen a different path leading
to the same (or different but also valid) solution.

If no solution can be found starting at the students current partial solution,
then either the path chosen by the student will indeed not lead to a successful
completion of the problem, or it will, but the algorithm is just not able to
find it.

When we have found a model solution, we track back the steps taken by the search
algorithm to provide a strategy that can be used to give hints to the student.
Ideally, if the type signature is strong enough, there is only one valid
solution to an Agda program.

\begin{itemize}
\item on the top level of the strategy, we can interleave the sub-strategies for
  solving individual clauses of a function definition. (they are independent of
  each other)
\item trying to solve a certain clause can produce more clauses when case splits
  are performed
\item when solving a clause, case splits must always be performed before
  refinements
\item the order of refining holes in the current clause generally does not
  matter (i.e.\ it could be interleaved), but in certain cases Agda will only
  accept a refinement for a function argument if another argument is refined
  before
\item the case split strategy is constructed from a valid order of splits that
  lead to a successful completion of the right hand side
\item the refinement strategy is guided by the steps taken by the proof search
\end{itemize}

\subsection{Examples}

\todo[inline]{elaborate on some examples by showing the search for a solution in
detail (i.e.\ including the unifications necessary)}

\paragraph{map function} A stepwise construction of a map function on vectors,
including the intermediate goal types.
\begin{enumerate}
\item The initial program
\begin{lstlisting}[mathescape, language = agda]
data Nat : Set where
  zero : Nat
  succ : Nat -> Nat

data Vec (a : Set) : Nat -> Set where
  Nil  : Vec a zero
  Cons : forall {n} -> a -> Vec a n -> Vec a (succ n)

map : forall {A B n} -> (A -> B) -> Vec A n -> Vec B n
map f xs = {!!}
\end{lstlisting}
The type of the hole is \lstinline[language=agda]|Vec B n| where A, B and n are
implicit. The argument types are \lstinline[language=agda]|f : A -> B| and
\lstinline[language=agda]|xs : Vec A n|. Searching for refinements using only
definitions in scope would lead to the tautological definition
\lstinline[language=agda]|map f xs|, which will be rejected because the
arguments are not structurally smaller.

\item The only parameter that can be split is \texttt{xs}

\end{enumerate}

\section{Formalization}

In this section, we formalize the subset of Agda we are interested in for
deriving automatically a program that satisfices some type.

\subsection{Patterns}

A pattern is a tree like structure that may be either a variable or a type
constructor applied to other patterns. When case splitting we should be able to
do so either in a top level variable or a possibly nested variable inside the
pattern. As an invariant we know that the variables appearing all along the
patterns are unique.

\begin{equation}
  \begin{array}{rcl}
    x & \in & \mathbf{Var} \\
    p & ::= & x \mid \mathbf{Constr}\ p_1\ \ldots\ p_n
  \end{array}
\end{equation}

We define the following function that given a pattern it returns the set of
variables that it contains.

\begin{equation}
  \begin{array}{rcl}
    vars(x) & = & \{x\} \\
    vars(\mathbf{Constr}\ p_1\ \ldots\ p_n) & = & \bigcup_{i=1}^{n} vars(p_i)
  \end{array}
\end{equation}

Another important consideration is that we need a function that allows us to
replace a pattern variable by a new pattern in case we are case splitting in
such variable. Of course it is not straightforward as variables in patterns
might be nested withing type constructors.

\begin{equation}
  \begin{array}{rcl}
    replace(x,p,y) & = & y\\
    replace(x,p,x) & = & p\\
    replace(x,p,\mathbf{Constr}\ p_1\ \ldots\ p_n) & = & \mathbf{Constr}\
    replace(x,p,p_1) \ldots replace(x,p,p_n)
  \end{array}
\end{equation}

\subsection{Type}

% The types we consider are either $Set$, D_i

\subsection{Expression}

An expression is part of the right hand side of a function clause. We consider
as expressions either a variable possibly applied to a number of other
expressions, or a type constructor applied to a list of expressions or a hole
that we denote with \textbf{?}.

It is important to note that unlike raw Agda, we do not consider the posibility
that expressions may be themselves lambda abstractions which would lead us to
higher order unification during proof search which is known to be undecidable.
In the case an auxiliary function is needed it must be already avaliable from
another top level binding.

\begin{equation}
  \begin{array}{rcl}
    e & ::= & x\ e_1 \ldots e_n\ (x \in \mathbf{Var})\ \mid\ \mathbf{Constr}\
    e_1\ \ldots\ e_n\ \mid\ \textbf{?}
  \end{array}
\end{equation}

We call an expression \textit{complete} if it does not have any hole \textbf{?}
remaining either in the top level or in any of its subexpressions. We can define
a function that computes this.

\begin{equation}
  \begin{array}{rcl}
    complete?(\textbf{?}) & = & \textbf{F} \\
    complete?(\mathbf{Constr}\ e_1\ \dots\ e_n) & = &\textbf{T}\ \bigwedge_{i=1}^{n}
    complete?(e_i)\\
    complete?(x\ e_1\ \ldots\ e_n) & = & \textbf{T}\ \bigwedge_{i=1}^{n}
    complete?(e_i)
  \end{array}
\end{equation}

\subsection{Clause}

A function definition is just a set of clauses that relate a set of patterns
with an expression.

If any of the patterns in the left hand side is a type constructor, then there
should be a function clause for every other possible (well typed) constructor of
the type so the full type is covered by the clauses. We will suppose from now on
that this is the case.

In order to facilitate the formalization, we are going to explicitly type every
pattern and the resulting expression. Note that in this case we omit possible
dependencies on the types given other patterns. This will be made again explicit
when we define the step relation.

\begin{equation}
  \begin{array}{rcl}
    c & ::= & (p_1\ :\ T_1)\ (p_2\ :\ T_2)\ \dots\ (p_n\ :\ T_n)=\ (e : T_{n+1})\ \mid\
    \bigwedge_{i=1}^{n} c_i
  \end{array}
\end{equation}

We also define a function that can inspect a clause and decide whether or not we
have reached a complete definition. We call a clause \textit{complete} when the right
hand side expression is \textit{complete}.

\begin{equation}
  \begin{array}{rcl}
    complete?((p_1 : T_1)\ (p_2 : T_2) \dots (p_n : T_n)=\ (e :
    T_{n+1})) & = & complete?(e)\\
    complete?(\bigwedge_{i=1}^{n} c_i) & = & \bigwedge_{i=1}^{n} complete?(c_i)
  \end{array}
\end{equation}

\subsection{Step relation $\leadsto$}

We can now give the full definition of what a step means. We are going to
characterize a step as a relation on \textbf{Clause}. We have to include in this
relation a signature $\Sigma$ that contains the relevant information for types
and a context $\Gamma$ of any other relevant functions that may be avaliable.

This relation is not deterministic. At each step we can choose to try the proof
search algorithm on the hole in the right hand side, or perform a case split on
some variable in the left hand side of the clause. We restric as Agda the case
split only to the case where the full expression in the right hand side is still
a hole and has not been partially fullfiled.

Now we can give the first rule that relates two clauses where in the second one
the hole has been succesfully filled by an expression found by the proof search
algorithm.

\begin{figure}[h]
  \centering
  \footnotesize
  \AxiomC{\textit{proof search}$(\Sigma, extend(\Gamma,p) , T_{n+1}\ p_1\dots p_n) = e$}
  \AxiomC{$complete?(e)$}
  \RightLabel{Step\_Proof\_Search}
  \BinaryInfC{$\Sigma ; \Gamma \vdash (p_1:T_1) \dots (p_n : T_n\ p_1 \dots p_{n-1})\
  =\ (\textbf{?}:T_{n+1}\ p_1\dots p_n) $}
  \noLine
  \UnaryInfC{$\leadsto$}
  \noLine
  \UnaryInfC{$\Sigma ; \Gamma \vdash
  (p_1:T_1) \dots (p_n : T_n\ p_1 \dots p_{n-1})\
=\ (e:T_{n+1}\ p_1\dots p_n)$}
  \DisplayProof
\end{figure}

The second rule that we need to define is the one that relates two clauses when
the actual case split is performed. In this case is not just a rule, but a rule
schema that can be instatiated for every possible variable on where we can
perform the splitting. This is denoted as a subscribe in $\leadsto_{k:x}$.

\begin{figure}[h]
  \centering
  \footnotesize
  \AxiomC{\textit{constructors}$(\Sigma, T_k\ p_1 \dots p_{k-1}) = \mathbf{Constr}_1 \dots \mathbf{Constr}_m$}
  \AxiomC{$x \in vars(p_k)$}
  \AxiomC{$x_1 \dots x_j \in \textbf{fresh}$}
  \RightLabel{Step\_Case\_Split}
  \TrinaryInfC{$\Sigma ; \Gamma \vdash
    (p_1:T_1) \dots (p_k : T_k\ p_1 \dots p_{k-1}) \dots (p_n : T_n\ p_1 \dots
    p_{n-1})\ =\ (\textbf{?}:T_{n+1}\ p_1\dots p_n)$}
  \noLine
  \UnaryInfC{$\leadsto_{k:x}$}
  \noLine
  \UnaryInfC{$\Sigma ; \Gamma \vdash \bigwedge_{i=1}^{m} (\dot{p_1} : T_1) \dots
  (replace(p_k, x, \mathbf{Constr}_i\ x_1 \dots x_j) : T_i\ {p_1} \dots {p_{i-1}})
    \dots (\dot{p_n} : T_n\ \dot{p_1} \dots
    \dot{p_{n-1}})\ =\ (\textbf{?}:T_{n+1}\ \dot{p_1}\dots \dot{p_n})$}
  \DisplayProof
\end{figure}

In this rule, when we try to pattern match on a variable,. The resulting clause
is a compound of the original clause where the variable under consideration has
been replaced by each one of the constructors of the type. Moreover, because we
are dealing with dependent types it is important to note that pattern matching
on a variable of a specific type might influence as well other variables in
other patterns. We denote this by putting a dot over the patterns in the outcome
clause.

For the sake of completeness we also allow a step to take place in any of the
single clauses that may appear within a compound clause.

\begin{figure}[h]
  \centering
  \footnotesize
  \AxiomC{$c_i \leadsto c'_i$}
  \UnaryInfC{$\bigwedge c_1 \dots c_i \dots c_n \leadsto \bigwedge c_1 \dots c'_i \dots c_n $}
  \DisplayProof
\end{figure}

Now we are able to say that a stepwise construction of a program is the 

% \begin{figure}
%   \AxiomC{\textit{constructors}$(\Sigma, T_i\ p_1 \dots p_{i-1}) = C_1 \dots C_m$}
%   \AxiomC{$x \in vars(p_i)$}
%   \RightLabel{Step\_Case\_Split}
%   \UnaryInfC{$\Sigma ; \Gamma \vdash (p_1:T_1) \dots (p_i : T_i p_1 \dots
%   p_{i-1}) \dots (p_n : T_n\ p_1 \dots p_{n-1})\
%   =\ (\textbf{?}:T_{n+1}\ p_1\dots p_n) \leadsto_{x} (p_1:T_1) \dots (p_n:T_n\ p_1 \dots p_{n-1})\
% =\ (\textbf{?}:T_{n+1}\ p_1\dots p_n)$} \wedge
%   \DisplayProof
% \end{figure}


% a function is a pair of a type and a set of clauses
% f = (t , cl)

% a clause is of the form p_1 \dots p_n = e
% we define the valid operations as a relation on clause -> {clause}

% fn p_1 \dots p_n = ? \rightarrow {f
% $fun : (x_1 : T_1) \rightarrow (x_2 : T_2\ x_1) \rightarrow \cdots \rightarrow T_{n+1}\ x_1\ \ldots\ x_n$

% $\Sigma ; \Gamma \vdash (p_1\ :\ T_1)\ (p_2\ :\ T_2\ p_1)\ \dots\ (p_n\ :\ T_n\ p_1\ \dots\ p_{n-1})\
% =\ (\textbf{?}\ :\ T_{n+1}\ p_1\ \ldots\ p_n) \leadsto \{ p_1 \dots c \dots p_n = ?
%   | c \in C \} $

% We define a relation between clauses that would serve as a basis for generating
% a step wise based strategy for solving the problem.

% The relation we define $\leadsto$ is not deterministic as the


% \begin{figure}
%   \AxiomC{}
%   \RightLabel{T\_Env\_Empty}
%   \UnaryInfC{$\Sigma ; \Gamma \vdash \bullet$}
%   \DisplayProof
% \end{figure}

% \begin{figure}
%   \AxiomC{$\Sigma ; \Gamma, x_1 : T_1,\ x_2 : T_2\ x_1\ \dots\ x_n : T_n\ x_1
%   \dots x_{n-1} \vdash \bullet$}
%   \RightLabel{T\_Env\_Extend}
%   \UnaryInfC{$\Sigma ; \Gamma,\ x_1 : T_1,\ x_2 : T_2\ x_1\ \dots\ x_n : T_n\ x_1
%   \dots x_{n-1},\ x_{n+1} : T_{n+1}\ x_1 \dots x_n \vdash \bullet$}
%   \DisplayProof
% \end{figure}



% \begin{figure}
%   \AxiomC{}
%   \RightLabel{I-Constr}
%   \UnaryInfC{$p_1 \dots p_i \dots p_n = ? \leadsto \{ p_1 \dots p_i \dots p_n =  $}
%   \DisplayProof
% \end{figure}

% \begin{figure}
%   \AxiomC{}
%   \RightLabel{I-Fun}
%   \UnaryInfC{$p_1 \dots p_i \dots p_n = ? \leadsto \{ p_1 \dots p_i \dots p_n =  $}
%   \DisplayProof
% \end{figure}

% \begin{figure}
%   \AxiomC{}
%   \RightLabel{T-TFunc2}
%   \UnaryInfC{$p_1 \dots p_i \dots p_n = ? \leadsto \{ p_1 \dots c \dots p_n = ?
%   | c \in C \} $}
%   \DisplayProof
% \end{figure}

\bibliographystyle{plain}
\bibliography{literature}

\end{document}
