\documentclass[11pt,a4paper]{scrartcl}
\usepackage[utf8]{inputenc}
\usepackage[english]{babel}
\usepackage{geometry}
\usepackage{amsmath}
\usepackage{amssymb}
\usepackage{amsthm}
\usepackage{xcolor}
\usepackage{graphicx}
\usepackage{todonotes}
\usepackage{enumitem}  % http://www.ctan.org/pkg/enumitem
\usepackage[noabbrev]{cleveref}
\usepackage{fancyvrb}
\usepackage{unicornsoflove}
\usepackage{hyperref}
\usepackage{bussproofs}

 \DefineVerbatimEnvironment
   {code}{Verbatim}
   {} % Add fancy options here if you like.

\author{Carlos Tomé Cortiñas
  \and Fabian Thorand
  \and Ferdinand van Walree
  \and Renate Eilers
  \and  \small{Department of Information and Computing Sciences, Utrecht University}
  }
\title{Stepwise Construction of Simple Agda Programs}
\subtitle{Research Report}
\date{\today}

\begin{document}
\maketitle

\begin{abstract}
\end{abstract}

\tableofcontents

\section{Introduction}

Over the past decades, \emph{intelligent tutoring systems} (ITSs) have proven themselves beneficial to the acquisition of procedural skills such as algebraic equation solving or programming language proficiency\cite{Anderson1995}. ITSs, like for instance Ask-Elle\cite{Gerdes2012} for the Haskell programming language, allow users to request real-time feedback on their progress, and query the system for hints on how to continue. Observed successes of ITSs in learning programming languages\cite{Anderson1989},\cite{corbett1988problem} inspired the idea to create such a system for the dependently typed programming language and theorem prover Agda.

The aim of this research is to lay the building blocks for creating this system. Rather than relying on teachers to supply solutions to exercises, it would be more convenient for the tutor to be capable of calculating solutions for a given specification independently. This alludes to the following research question: \emph{What (sub-)class of semi-decidable programs (or, equivalently, proofs) can be automatically constructed in the programming language Agda through a series of mechanical steps that would normally be performed by a user writing a program, and how can we automatically extract these steps for programs within these classes?} 

The steps towards answering this question can be found in recent literature on automatic program synthesis and proof search. The work presented in this report reaches a solution in the form of a prototypical implementation of an intelligent tutoring system. This prototype is capable of generating model solutions for a given function specification (if any exist), and subsequently guiding a user to a solution in a stepwise manner. 

In the following we will discuss current research in the area of automatic program construction, and explain how these works relate to finding an answer the question posed earlier. We show how an existing algorithm for proof search can be used and extended to implement an interactive programming tutor, and shed a light on the capabilities and limitations of this implementation. We assess its results and discuss future work.

\section{Method}

Given a function specification, generating an Agda program is primarily a process of pattern matching on the right variables and supplying the correct refinements for holes that are left open. At any point, the type of a hole is determined. This makes proof search, or program construction per the Curry-Howard correspondence, a type-driven activity. A survey of the literature(\cite{Foster2011},\cite{Lindblad2006},\cite{Kokke2015}) shows that the subject of automatic, type-driven proof search is well-explored. A promising starting point for the specific purpose of this research was found in \cite{Kokke2015}, which demonstrates a type-driven proof search algorithm based on first-order unification for automatically filling holes using a set of definitions provided as hints. The algorithm permits users to bound search depth so as to allow searching for possibly semi-decidable programs whilst still ensuring termination. It should be noted that in the current version, finding solutions for goals involving function types is not possible. Additionally, it does not involve case splitting in its proof search. 

As pattern matching tends to take up an important part of program construction by human users, it is essential that the programming tutor incorporates this tactic. The system presented here uses the proof search algorithm mentioned above as a subroutine for hole refinement. If no such refinement can be found, it picks a variable to split on and subsequently searches for a way to refine the holes in the clauses that resulted from the case split. We shall now proceed to formalise these steps. Doing so requires the introduction of some definitions:

\subsection{Programs}

We define program to be of a set of functions, where functions consist of a type declaration and a set of clauses. The programming tutor expects a set of necessary auxiliary functions to be fully defined. All function types must be specified. This includes the function for which a strategy is to be generated.

\subsection{Clauses}

A clause relates a set of patterns on the left hand side with an expression on the right hand side. To facilitate formalization, every pattern and expression will be explicitly typed. Possible dependencies on types given other patterns are omitted. They will be made explicit again when the step relation is defined. It should be noted that all types are assumed to be in \texttt{Set}, the lowest level of the universe hierarchy.

\begin{equation}
  \begin{array}{rcl}
    c & ::= & (p_1\ :\ T_1)\ (p_2\ :\ T_2)\ \dots\ (p_n\ :\ T_n)=\ (e : T_{n+1})\ \mid\
    \bigwedge_{i=1}^{n} c_i
  \end{array}
\end{equation}

If any of the patterns on the left hand side are a type constructor, then there should be a function clause for every other possible (well typed) constructor of that type as well to make sure all cases are covered by the clauses. We will suppose from now on that this is the case.

For program construction it is import to know whether a complete definition is reached. A clause is \textit{complete} if its right hand side expression is \textit{complete}:

\begin{equation}
  \begin{array}{rcl}
    complete?((p_1 : T_1)\ (p_2 : T_2) \dots (p_n : T_n)=\ (e :
    T_{n+1})) & = & complete?(e)\\
    complete?(\bigwedge_{i=1}^{n} c_i) & = & \bigwedge_{i=1}^{n} complete?(c_i)
  \end{array}
\end{equation}

\subsection{Patterns}

A pattern is a tree-like structure that can be a variable, a type-constructor applied to other patterns, or an inaccessible pattern (also known as a \emph{dotted pattern}, as it is denoted by a pattern prefixed with `.'):

\begin{equation}
  \begin{array}{rcl}
    x & \in & \mathbf{Var} \\
    p & ::= & x \mid \mathbf{Constr}\ p_1\ \ldots\ p_n \mid .p
  \end{array}
\end{equation}

Case splitting can be performed on both top level variables and variables nested inside an accessible pattern. As an invariant, variables appearing along accessible patterns are unique. Inaccessible patterns may only reference variables bound in accessible patterns\cite{norell:thesis}.

The following function returns the set of variables that a given pattern contains:

\begin{equation}
  \begin{array}{rcl}
    vars(x) & = & \{x\} \\
    vars(\mathbf{Constr}\ p_1\ \ldots\ p_n) & = & \bigcup_{i=1}^{n} vars(p_i)
  \end{array}
\end{equation}

\subsection{Expressions}

Expressions can be either a variable, possibly applied to a number of other expressions, a type constructor applied to a list of expressions, or a hole, denoted by `\textbf{?}':

\begin{equation}
  \begin{array}{rcl}
    e & ::= & x\ e_1 \ldots e_n\ (x \in \mathbf{Var})\ \mid\ \mathbf{Constr}\
    e_1\ \ldots\ e_n\ \mid\ \textbf{?}
  \end{array}
\end{equation}

It is important to note that unlike raw Agda, we do not consider the option that expressions may be lambda abstractions, as this would require higher order unification during proof search, which is known to be undecidable. If an auxiliary function is used in an expression, it must have been declared in a top-level binding beforehand.

An expression is considered \textit{complete} if no holes remain in either the top level or in any of its sub-expressions. The following function computes this:

\begin{equation}
  \begin{array}{rcl}
    complete?(\textbf{?}) & = & \textbf{F} \\
    complete?(\mathbf{Constr}\ e_1\ \dots\ e_n) & = &\textbf{T}\ \bigwedge_{i=1}^{n}
    complete?(e_i)\\
    complete?(x\ e_1\ \ldots\ e_n) & = & \textbf{T}\ \bigwedge_{i=1}^{n}
    complete?(e_i)
  \end{array}
\end{equation}

\subsection{The Step-Relation} 

Now that all necessary definitions have been made, it is time to formalise precisely what a \emph{step} is. A step is characterised as a relation on clauses. Additionally, this relation involves a signature $\Sigma$ that contains relevant type declarations, and a context $\Gamma$ containing any other relevant functions that have been priorly declared.

The step relation is not deterministic. At each step we can choose to try the proof search algorithm on the hole in the right hand side, or perform a case split on some variable in the left hand side of the clause. It should be noted that a case split can only be performed on clauses where the right hand side is a hole.

The first rule for relating two clauses describes the case where a hole has been successfully refined by an expression found by the proof search algorithm:

\begin{figure}[h] 
  \centering
  \footnotesize
  \AxiomC{\textit{proof search}$(\Sigma, (p_1:\dots(p_n:\Gamma)\dots) , T_{n+1}\ p_1\dots p_n) = e$}
  \AxiomC{$complete?(e)$}
  \RightLabel{Step\_Proof\_Search}
  \BinaryInfC{$\Sigma ; \Gamma \vdash (p_1:T_1) \dots (p_n : T_n\ p_1 \dots p_{n-1})\
  =\ (\textbf{?}:T_{n+1}\ p_1\dots p_n) $}
  \noLine
  \UnaryInfC{$\leadsto$}
  \noLine
  \UnaryInfC{$\Sigma ; \Gamma \vdash
  (p_1:T_1) \dots (p_n : T_n\ p_1 \dots p_{n-1})\
=\ (e:T_{n+1}\ p_1\dots p_n)$}
  \DisplayProof
\end{figure}

The second rule describes the case split step. Intuitively, pattern matching on a variable results in a compound clause consisting of as many conjuncts as there are constructors for the type of the variable under consideration. Every conjunct is identical to the original, with the exception of the variable that is being split on: this has been replaced by each of the constructors of the variable's type. 

However, the actual step for case splits is a bit more involved because we have to take into account that Agda allows users to define inductive type families. Pattern matching on a variable might influence both other patterns whose type depends on this variable as an index, or types that are indices to such variable. When we match on a pattern of the inductive type family $\texttt{T i}_\texttt{1} \ldots \texttt{i}_\texttt{n} $ with a constructor of the type $\texttt{C : (x}_\texttt{1}\texttt{ : A}_\texttt{1}\texttt{)} \rightarrow \ldots \rightarrow \texttt{(x}_\texttt{m}\texttt{ : A}_\texttt{m}\texttt{)} \rightarrow \texttt{T j}_\texttt{1} \ldots \texttt{j}_\texttt{n}$ Agda will try to unify every $\texttt{i}_\texttt{i}$ with  $\texttt{j}_\texttt{i}$. If the resulting unification causes some left hand side variable \texttt{x} to be instantiated with a value \texttt{v}, then \texttt{x} will be replaced by \texttt{.v} in the left hand side of the resulting clause. 

The case split rule is not just a rule, but a rule schema that can be instantiated for every variable on which a split can be performed. The selected variable is denoted by \textit{x} in $\leadsto_{x}$. Note that in the rule below, $LHS$  is an abbreviation for $(p_1:T_1) \dots (p_n : T_n\ p_1 \dots p_{n-1})$.

\begin{prooftree}
  \def\defaultHypSeparation{\hskip.05in}
  \centering
  \footnotesize
  \AxiomC{$x \in vars(LHS)$}
  \AxiomC{$\Sigma \vdash x : T_x$}
  \AxiomC{\textit{constructors}$(T_x) = \mathbf{C}_1 : T_{C1}, \dots, \mathbf{C}_m : T_{Cm}$}
  \AxiomC{$unify(\Sigma,\Gamma,T_{Ci},T_x) = \sigma_i$}
  \RightLabel{Step\_Case\_Split}
  \QuaternaryInfC{$\Sigma ; \Gamma \vdash 
    LHS =\ (\textbf{?}:T_{n+1}\ p_1\dots p_n)$}
  \noLine
  \UnaryInfC{$\leadsto_{x}$}
  \noLine
  \UnaryInfC{$\Sigma ; \Gamma \vdash \bigwedge_{i=1}^{m} LHS\{ \sigma_i \circ [x \mapsto C_i] \} \ =\ (\textbf{?}:T_{n+1}\ \dot{p_1}\dots \dot{p_n})$}
\end{prooftree}

For completeness, we also include a rule that allows a step to take place in any of the single clauses that may appear within a compound clause.

\begin{figure}[h]
  \centering
  \footnotesize
  \AxiomC{$\Sigma ; \Gamma \vdash c_i \leadsto c'_i$}
  \UnaryInfC{$\Sigma ; \Gamma \vdash \bigwedge c_1 \dots c_i \dots c_n \leadsto \bigwedge c_1 \dots c'_i \dots c_n $}
  \DisplayProof
\end{figure}

\section{Implementation}

In order to answer the research question, we have developed a prototype implementation\footnote{The code can be found in \url{https://github.com/carlostome/martin}.} that we named Martin.\footnote{After the Swedish logician and philosopher Per Martin-Löf who developed the theory underlying Agda.} We have implemented Martin using the programming language Haskell. Since Agda is also implemented in Haskell, we can seamlessly interface with it.

The main component of Martin, is a library that provides the required functionality to automatically generate step wise strategies for completing Agda programs. Moreover, the library also provides an API to develop intelligent tutor systems for Agda. In order to demonstrate how easily the Martin library can be used for building applications we include a graphical application withing the package.

In this section, we briefly give an overview of the actual implementation of the library. 

\subsection{Strategy}

In the previous section, we formalized what a step should consist of when generating a strategy for an Agda program. A strategy for an exercise consists of one \texttt{ClauseStrategy} (as shown below) for every hole  in the exercise module.

\begin{lstlisting}[mathescape, language = haskell]
data ClauseStrategy
  = SplitStrategy String [ClauseStrategy]
  | RefineStrategy Proof
\end{lstlisting}

A clause can by solved by refinement (\texttt{RefineStrategy}) if we have a proof generated by the proof search. Alternatively, we can split on a variable (\texttt{SplitStrategy}) and apply a strategy to each of the resulting clauses.

The core of our strategy construction is implemented in the \texttt{Martin.Strategy} module by the two functions.
\begin{lstlisting}[mathescape, language = haskell]
proofSearchStrategy :: StatefulProgram 
                     -> InteractionId -> Search ClauseStrategy
splitStrategy :: StatefulProgram 
               -> InteractionId -> Search ClauseStrategy
\end{lstlisting}
A \texttt{StatefulProgram} consists of the AST of an Agda program coupled with the state of the type checker after successfully checking it. We need to store snapshots of Agda's state because almost every operation in the Agda monad is inherently stateful and we often need to revert that state to known checkpoints.

The \texttt{proofSearchStrategy} tries to find a refinement using the proof search algorithm. If that fails while still being in a top-level hole, it invokes the \texttt{splitStrategy} instead.
Our \texttt{Search} monad provides the exploration of multiple alternatives by means of the \texttt{MaybeT} transformer.

\subsection{User Interface}

We also implemented a rudimentary terminal user interface using the \texttt{brick-0.11}\footnote{It is available on Hackage under \url{https://hackage.haskell.org/package/brick-0.11}.} library. It presents the source code and provides means for a student to interact with holes.

On the one hand, this involves requesting contextual information, such as the type of the goal or the definitions in context as well as getting more specific hints based on the computed strategy.
Our hints are partitioned in levels revealing more and more details if the student repeatedly asks.

On the other hand, a student can perform a refinement or a case-split in a hole. The action is then compared to the strategy and the user is notified if his action diverges from the strategy, and if so, in what regard.
Additionally, a new strategy is computed based on the new state of the program. The user is notified if the tutor was unable to generate a strategy from their actions.

\section{Results}
Having implemented the intelligent programming tutor Martin, we can now focus on determining the different classes of programs that can be automatically constructed. We will do this by showing how Martin can derive solutions for Agda programs that differ in how dependent types are used and in how many example properties the program requires. Furthermore, we will also discuss what classes of programs cannot be automatically constructed by Martin.
\begin{lstlisting}[mathescape, language = agda]
map : forall {A B n} -> (A -> B) -> Vec A n -> Vec B n
map f vnil = vnil
map f (vcons x xs) = vcons (f x) (map f xs)
\end{lstlisting}
Martin was able to completely derive the solution to the function map that maps a function over a vector whose length is encoded in the \texttt{Vec} data type. If the Vector is empty, then the type signature restricts the result to be an empty Vector as well. Martin's proof search then unifies the goal, the empty vector type, to the \texttt{vnil} rule and therefore refines the hole with \texttt{vnil}. Likewise for a non-empty vector, it will derive a refinement with \texttt{vcons}, since the resulting vector must also be non-empty. The remaining two holes must be an element of type \texttt{B} and a vector containing elements of type B of length one less than the argument vector. Also for the element type, Martin's proof search was able to find correct refinements. Even though this is a very simple program, Martin was able to derive a correct solution using only the type of the elements in the vector and the length of the vector.

\begin{lstlisting}[mathescape, language = agda]
append : forall {a b n} -> Vec A m -> Vec A n -> Vec A (add m n)
append vnil ys = ys
append (vcons x xs) ys = vcons x (append xs ys)
\end{lstlisting}
Another example where Martin is able to derive a solution using dependent types is the append function. We apply add to the given vector lengths and state that the resulting vector must have that size. The only way for Martin to construct a program that satisfies the result type, given the local variables and defined functions, is to actually append the first vector to the second vector. Whilst this is mostly a feature of dependent types, we will see later that Martin can't derive all dependently typed programs. 
 
\begin{lstlisting}[mathescape, language = agda]
append : forall {A} -> List A -> List A -> List A
append nil ys = ys
append (cons x xs) ys = cons x (append xs ys)

appendP1 : forall {x l l'} -> append (cons x l) l' == cons x (append l l')
appendP1 = refl

appendP2 : forall {l} -> append nil l == l
appendP2 = refl
\end{lstlisting}
We will look at the append function once more, but this time we have replaced the vector data type by the list data type.
Since no length information is given, Martin does not have enough information to derive a program that is correct w.r.t.\ both types and semantics.
The simplest solution of the correct type is to always return the empty list.
To solve this problem, we can add example properties.
The types of these properties guide Martin to determine what correct and incorrect derivations are by influencing which solution candidates found by the proof search actually type-check.
For example, if we have case-split on the first list, then according to the type of append, nil would have been a valid result. However, the proposition \texttt{appendP2} states that if the first list is empty, then the result should be equal to the second list. Martin is now guided to the correct solution for this hole, which is \texttt{ys}.
However, the use of such equality based properties is rather limited. They just correspond to the two function clauses of the correct solution, and hence it is even more work to specify such properties than to just specify a model solution.

% Fabian: changed the following part; you don't need with for that.
% given a function 'or : Bool -> Bool -> Bool' and a function 'eq : Nat -> Nat -> Bool', you can define
% isMember x nil = false
% isMember x (cons y ys) = or (eq x y) (isMember x ys)
%
% Still, the type of isMember itself is just to weak, that's why I added the example below:

Nonetheless, instead of specifying external properties it is often possible to encode them directly in the type of the function being implemented.
The following code snippet shows a type signature of a function to check whether a particular element is part of a list.
\begin{lstlisting}[mathescape, language = agda]
isMember : Nat -> List Nat -> Bool
isMember x xs = ?
\end{lstlisting}
While such a function would certainly return a boolean value in conventional programming languages, we can (and have to) do better in Agda. This type signature does not give any guarantees about whether the element is indeed part of the list.
Furthermore, it does not provide useful information to the proof search algorithm, as any boolean value would fulfill the requirements.

A function that can actually be usefully applied in proofs would itself need to return a proof whether an element is part of a list or not. That is what \emph{decidable equality}, encoded in the \texttt{Dec A} type, is about. It can either be a proof \texttt{A}, or a proof that, given \texttt{A}, we can generate an inhabitant of the empty type. That ultimately means that the proof \texttt{A} could not have existed in the first place.

\begin{lstlisting}[mathescape, language = agda]
data Empty : Set where

data Not (A : Set) : Set where
  is-absurd : (A -> Empty) -> Not A

data Dec (A : Set) : Set where
  yes : A -> Dec A
  no : Not A -> Dec A
\end{lstlisting}

As Martin is not able to synthesize lambda abstractions, we wrap the negation of a proof in a data type, to hide it from the proof search algorithm. A proof that an element is part of a list now simply consists of a path through the cons constructors, encoded by the \texttt{Member} data type.
Since an implementation of \texttt{member} requires with-clauses which cannot (yet) be introduced by Martin, a skeleton of the function body containing all the required with-clauses has to be given.

\begin{lstlisting}[mathescape, language = agda]
data Member {A : Set} (x : A) : List A -> Set where
  here : {xs : List A} -> Member x (cons x xs)
  there : forall {y} -> {xs : List A} -> Member x xs -> Member x (cons y xs)

empty-no-member : forall {A} -> {x : A} -> Not (Member x nil)

neither-here-nor-there : forall {A} -> {x y : A} {ys : List A} -> Not (x == y) 
  -> Not (Member x ys) -> Not (Member x (cons y ys))

member : forall {A} -> ((x y : A) -> Dec (x == y)) -> (v : A) -> (vs : List A) -> Dec (Member v vs)
member eq? x nil = no empty-no-member
member eq? x (cons y ys) with eq? x y
member eq? x (cons y ys) | yes p = {!!}
member eq? x (cons y ys) | no pf with member eq? x ys
member eq? x (cons y ys) | no pf | member = {!!}
\end{lstlisting}
The remaining holes can now be solved automatically. This shows, that a stronger type usually also gives more useful results. However, the lemmas \texttt{empty-no-member} and \texttt{neither-here-nor-there} need to be supplied by the teacher.

% Fabian: I don't get the purpose of the following example. What difference does it make whether one uses
% a vector or a list to find whether an element is part of it?
% Edit: (Commented out that part, since we are already at the page limit}

%\begin{lstlisting}[mathescape, language = agda]
%isMemberV : forall {n} -> (x : Nat ) -> Vec Nat n -> Bool
%listToVec : (l : List Nat) -> Vec Nat (length l)
%\end{lstlisting}
%One way to get around this limitation, but only for specific programs, is to introduce a conversion from one data type to another data type. There should also exist a function isMember that works for the other data type. For this example, we have provided two auxillary functions. One for converting a list to a vector and another that implements isMember for vectors. 
%\begin{lstlisting}[mathescape, language = agda]
%isMember : Nat -> List Nat -> Bool
%isMember x xs = isMemberV x (listToVec xs)

%isMemberP1 : forall {n l} -> isMember n l == isMemberV n (listToVec l)
%\end{lstlisting}
%Then using a property we can guide Martin to convert the list to a vector and then apply isMemberV to this vector. Similar to append for lists, writing the property is as much work as defining the solution by hand. We also had to come up with two auxiliary functions, although the conversion function could be automatically derived.

The last limitation we will discuss is the restriction to first order unification.
Syntactically this means that variables cannot have arguments in proof search terms.
A common case where such a situation arises is the \emph{congruence} rule for equality,
typed in Agda as follows.
\begin{lstlisting}[mathescape, language = agda]
cong : forall {A B} -> (f : A -> B) -> {x y : A} -> x == y -> f x == f y
\end{lstlisting}
Being restricted to first-order unification, Martin is unable to generate a proof search rule for this function,
because the variable \texttt{f} is applied to an argument in the return value.
Nevertheless, it is able to find an implementation to the \texttt{cong} itself, because in the function body
all variables have been instantiated with constants. Pattern matching on the proof argument unifies \texttt{x ~ y}
and therefore reduces the goal type to \texttt{f x == f x}, which in turn can be unified with the type of \texttt{refl},
since \texttt{f} and \texttt{x} are constants in this scope.

Another limitation of our work is that is not possible to effectively generate strategies in the presence of input/output properties. An input/ouput property defines the relation between the input of a function and the output it $\beta$-reduces to. For example we could have the following i/o property,

\begin{lstlisting}[mathescape, language = agda]
prop1 : map id (cons zero nil) == cons zero nil
prop1 = refl
\end{lstlisting}

We might expect that given several input/output properties would at least serve to derive a strategy that satisfies those properties. However, our approach to case splitting assumes that every hole in a program is independent of each other and therefore a strategy for filling a hole can be also generated independently of other holes. When a hole is refined, it is type checked against the program with the other remaining holes, and in this way we are delaying the satisfiability of the properties to those other holes, leading to an incorrect strategy.

\section{Conclusion and Future Work} % saving space by heaving one headline less

This project has shown how we can leverage Agda's type system to automatically generate model solutions for programming exercises.
Furthermore, it is possible to extract a strategy that is used to give hints and feedback to students.
Compared to programming tutors like Ask-Elle for Haskell, where a teacher has to specify multiple model solutions, automatically generating these is an obvious advantage.
Moreover, if a student deviates from the initial strategy, our system can simply search for a new strategy starting at the students partial implementation.
In Ask-Elle, students leaving the path laid out by the model solution are on their own and no longer receive hints or feedback.
We have also improved over the version of automatic program construction presented in the paper Auto In Agda by being able to perform case splitting when the search procedure cannot find a suitable answer.

A significant part of our work was trying to understand the internal mechanics of the Agda compiler to be able to attach our system on top of it. 

Currently, our system faces several limitations. The most notable is the restriction to first-order unification during proof search and not being able to synthesize lambda terms. This can partly be mitigated by hiding the additional type complexity behind simpler data types.
While these auxiliary definitions cannot be proven themselves using our system, they can then at least be used in other proofs.
Additionally, some programs require matching on the results of (recursive) function calls. In Agda, this can be done using the \texttt{with} notation.
While our system is able to work with such clauses that are already present, it is, at the current state, unable to introduce such pattern matching itself.

Lastly, neither does it not support external properties (i.e.\ those that are not encoded in the relevant types) nor input/output examples.

% Extend with properties to prune search tree
% Heuristics for variable selection (for instance by expanding the variable with the biggest impact)
The work just presented here is basic and it leaves a lot of room for improvement. We, now, enumerate and briefly discuss what we think it can be interesting lines of work.
\begin{itemize}
\item Integration of Martin within the Ideas\cite{Heeren2010} framework.

It should be feasible to integrate our work with the Ideas framework, such that we regard an Agda program as the domain in hand, and any well-type preserving transformation as a step. As for any other domain, case splitting and refining will be regarded as the expert defined rules.

\item Support other dependently typed programming languages.

The work developed here is not exclusive of Agda but can be generalized until some extent to other dependently typed programming languages such as Idris\cite{brady2013idris}. Even in a near future, when Dependent Haskell\cite{DepHaskell} is fully developed, we might be able to apply there the same ideas.

\item Improve proof search to handle a restricted version of higher-order unification.

The proof search algorithm now only handles first order unification as higher order unification in general is not decidable. However, there has been some work on the area that restricting the kind of higher order components (functions mainly) that you can encode makes unification decidable. For more on this, please refer to the work of Miller\cite{Miller91alogic}.

\item Improve heuristics for case splitting

Correctly selecting a variable for case splitting is crucial to control the size of the search space when generating a strategy. In the dependently typed setting, several variables on the left hand side of a function clause can be related through the type they stand for. Case splitting on one of them allows us to gain more information about the others, and in many cases reduce the number of clauses needed to fully define a function. Variable dependencies within patterns induce a DAG where the dependency is encoded as an arc in the graph. 
A thorough analysis of the DAG should be able to determine the variable with the highest impact on other variables when case splitting.
However, it is not clear whether such variable has to appear as a leaf node or in some inner node.

\item Use properties within the proof search algorithm to prune the search space.

Currently the proof search algorithm has no context sensitive information that helps prune the search space when refining a hole. It would be interesting to see if somehow we can encode some kind of property and effectively use it.

% \item Study how to integrate input/output properties so the strategy generation is sound.

% In our approach we considered that the process of generating a strategy for filling more than one hole is to be taking as independently generating a strategy for each of the holes. However, if we want to use Agda type system to encode input/output relations involving the function(s) we are interested in, might lead to inconsistent results as the strategy generated for one hole might independently typecheck 

\end{itemize}


% Fabian: Do we really need an appendix?
% Carlos: No
% \section*{Appendix}





%%%%%%%%%%%%%%%%%%%%%%%%%%%%%%%%%%%%%%%%%%%%%%%%%%%%%%%%%%%%%%%%%%%%%
% Old stuff
% (To be worked in later)
%%%%%%%%%%%%%%%%%%%%%%%%%%%%%%%%%%%%%%%%%%%%%%%%%%%%%%%%%%%%%%%%%%%%%1

% Now we are able to say that a stepwise construction of a program is the 

% \begin{figure}
%   \AxiomC{\textit{constructors}$(\Sigma, T_i\ p_1 \dots p_{i-1}) = C_1 \dots C_m$}
%   \AxiomC{$x \in vars(p_i)$}
%   \RightLabel{Step\_Case\_Split}
%   \UnaryInfC{$\Sigma ; \Gamma \vdash (p_1:T_1) \dots (p_i : T_i p_1 \dots
%   p_{i-1}) \dots (p_n : T_n\ p_1 \dots p_{n-1})\
%   =\ (\textbf{?}:T_{n+1}\ p_1\dots p_n) \leadsto_{x} (p_1:T_1) \dots (p_n:T_n\ p_1 \dots p_{n-1})\
% =\ (\textbf{?}:T_{n+1}\ p_1\dots p_n)$} \wedge
%   \DisplayProof
% \end{figure}


% a function is a pair of a type and a set of clauses
% f = (t , cl)

% a clause is of the form p_1 \dots p_n = e
% we define the valid operations as a relation on clause -> {clause}

% fn p_1 \dots p_n = ? \rightarrow {f
% $fun : (x_1 : T_1) \rightarrow (x_2 : T_2\ x_1) \rightarrow \cdots \rightarrow T_{n+1}\ x_1\ \ldots\ x_n$

% $\Sigma ; \Gamma \vdash (p_1\ :\ T_1)\ (p_2\ :\ T_2\ p_1)\ \dots\ (p_n\ :\ T_n\ p_1\ \dots\ p_{n-1})\
% =\ (\textbf{?}\ :\ T_{n+1}\ p_1\ \ldots\ p_n) \leadsto \{ p_1 \dots c \dots p_n = ?
%   | c \in C \} $

% We define a relation between clauses that would serve as a basis for generating
% a step wise based strategy for solving the problem.

% The relation we define $\leadsto$ is not deterministic as the


% \begin{figure}
%   \AxiomC{}
%   \RightLabel{T\_Env\_Empty}
%   \UnaryInfC{$\Sigma ; \Gamma \vdash \bullet$}
%   \DisplayProof
% \end{figure}

% \begin{figure}
%   \AxiomC{$\Sigma ; \Gamma, x_1 : T_1,\ x_2 : T_2\ x_1\ \dots\ x_n : T_n\ x_1
%   \dots x_{n-1} \vdash \bullet$}
%   \RightLabel{T\_Env\_Extend}
%   \UnaryInfC{$\Sigma ; \Gamma,\ x_1 : T_1,\ x_2 : T_2\ x_1\ \dots\ x_n : T_n\ x_1
%   \dots x_{n-1},\ x_{n+1} : T_{n+1}\ x_1 \dots x_n \vdash \bullet$}
%   \DisplayProof
% \end{figure}



% \begin{figure}
%   \AxiomC{}
%   \RightLabel{I-Constr}
%   \UnaryInfC{$p_1 \dots p_i \dots p_n = ? \leadsto \{ p_1 \dots p_i \dots p_n =  $}
%   \DisplayProof
% \end{figure}

% \begin{figure}
%   \AxiomC{}
%   \RightLabel{I-Fun}
%   \UnaryInfC{$p_1 \dots p_i \dots p_n = ? \leadsto \{ p_1 \dots p_i \dots p_n =  $}
%   \DisplayProof
% \end{figure}

% \begin{figure}
%   \AxiomC{}
%   \RightLabel{T-TFunc2}
%   \UnaryInfC{$p_1 \dots p_i \dots p_n = ? \leadsto \{ p_1 \dots c \dots p_n = ?
%   | c \in C \} $}
%   \DisplayProof
% \end{figure}

\bibliographystyle{plain}
\bibliography{literature}

\end{document}
